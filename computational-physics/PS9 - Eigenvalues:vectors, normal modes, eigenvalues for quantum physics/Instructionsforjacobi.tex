Dear Physics 40 class,

On the course website you will find five files needed for today's lab:
jacobi.c           the subroutine for diagonalizing a matrix
nrutil.c           a file containing some utilities used by jacobi.c
nrutil.h           a header file for nrutil.c
jacobi_test.c      the main module which reads in the matrix and
                   calls jacobi.c
input.txt          sample input file

Here's how to use them:

First, compile the different pieces:

gcc -o jacobi_test.o -c jacobi_test.c
gcc -o jacobi.o -c jacobi.c
gcc -o nrutil.o -c nrutil.c

The '-o' tells the compiler to create object files with the indicated
names.

Then put everything together:

gcc -o jacobi_test jacobi_test.o jacobi.o nrutil.o -lm

Then run the program, which is called jacobi_test (again, the '-o' 
told the compiler to give it this special name instead of the default
'a.out')

jacobi_test

You can either input stuff from the screen:  first the dimension of the
matrix, then the entries in each row, with each entry separated by a
space
and the rows separated by a carriage return, or you can enter the same 
information into a file, and then tell the program to look in the file
for the input via the command '< input.txt'.  I believe it is easier
to use the latter approach.

jacobi_test < input.txt

Here's what should happen:

 $ jacobi_test
dimension of the matrix: 5
enter a 5 x 5 matrix (separated by space):
3 0 1 0 .5
0 4 1 0 .1
1 1 5 .4 .2
0 0 .4 2 1
.5 .1 .2 1 3

eigen problem for matrix A:
  3.000  0.000  1.000  0.000  0.500
  0.000  4.000  1.000  0.000  0.100
  1.000  1.000  5.000  0.400  0.200
  0.000  0.000  0.400  2.000  1.000
  0.500  0.100  0.200  1.000  3.000
 
number of Jacobi applied: 49
eigenvalues:
  2.523  3.834  5.982  1.280  3.382

eigenvectors:
  0.820 -0.391  0.305  0.234 -0.168
  0.257  0.689  0.426  0.071  0.523
 -0.376 -0.062  0.827 -0.140 -0.388
 -0.346 -0.301  0.124  0.804  0.357
 -0.030 -0.528  0.162 -0.523  0.648


